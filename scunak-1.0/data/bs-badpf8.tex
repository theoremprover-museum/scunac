In order to give a sample proof, we shall prove the first equation in (d).
Let $x$ be an element of $A\cap (B\cup C)$, then $x\in A$ and $x\in
B\cup C$.
Hence we either have (i) $x\in A$ and $x\in B$, or we have (ii) $x\in A$ and
$x\in C$. Therefore, either $x\in A\cap B$ or $x\in A\cap C$, so $x\in
(A\cap B) \cup (A\cap C)$.  This shows that $A\cap (B\cup C)$ is a
subset of $(A\cap
B)\cup (A\cap C)$.\\

Conversely, let $y$ be an element of $(A\cap B)\cup (A\cap C)$. Then,
either (iii) $y\in A\cap B$, or (iv) $y\in A\cap C$.  It follows that
$y\in A$, and either $y\in B$ or $y\in C$. Therefore, $y\in A$ and $y\in
B\cup C$ so that $y\in A \cap (B\cup C)$. Hence $(A\cap B) \cup
(A\cap C)$ is a subset of
$A\cap (B\cup C)$.\\

In view of Definition 1.1.1, we conclude that the sets $A\cap (B\cup C)$
and $(A \cap B) \cup (A\cap C)$ are equal.  
